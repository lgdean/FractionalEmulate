\documentclass[11pt]{article}
\usepackage{fullpage}
\usepackage{parskip}
\usepackage{tex-squares}
%\usepackage{doublespace}
%\usepackage{graphicx}
%\usepackage{moreverb}

\title{The Fractional Emulate Concept (DRAFT)}
\author{Rachel Chasin and Laura Dean}
\date{\today}
%\markright{foo}


\begin{document}
\thispagestyle{empty}
\maketitle

%\tableofcontents

In this paper we describe the fractional emulate concept, invoked
with $<$fraction$>$ emulate $<$call$>$. This concept
is analogous to fractional stable: do $<$call$>$ until you've
turned $<$fraction$>$, and then continue with the rest of the
call, but emulate it -- that is, do any subsequent turning motions 
but remain in place.

As with the fractional stable concept, different people may start
working emulate at different times, although unlike with fractional
stable, this may result in an illegal call (see
Section~\ref{sec:nonanalogy}).

The motivation for this concept came about when Laura expressed
frustration that Zip the Top had no apparent relation to the other
``the top'' calls -- in particular, nobody was casting off 3/4.
``Well, actually, it's Fan the Top, but emulate after turning 1/4,''
Janet helpfully noted.  And so it is!
Rachel noticed that it oughtta have a name (and named it).

Some more fun equivalences:

\begin{tabular}{|l|l|}
\hline
1/4 emulate Fan the Top   & Zip the Top \\
1/4 emulate Cast off 3/4 (from mini-wave)  & Shazam \\
1/4 emulate Box Transfer & Follow Thru and U-Turn Back \\
& (tandem twosome Shazam) \\
1/4 emulate Follow Your Neighbor  & Same as previous\\
1/4 emulate Trade                          & Single Wheel \\
1/4 emulate Scoot Back  & Single Ferris Wheel \\
& (tandem twosome Single Wheel)\\
1/4 emulate Couples Trade  & Stack the Wheel \\
1/4 emulate Box Counter Rotate 1/2 & Couple Up \\
1/4 emulate Flip the Line ($n$/4) & Patch the centers \\
%(but not 1/4 emulate 2/3 Recycle) & \\
\hline
\end{tabular}

%Fractional emulate feels appropriate for describing calls
%involving a U-Turn Back -- for example, Zip the Top and Shazam.
%TODO: still need something to lead into this sentence?
Likewise, Follow Your Neighbor involves substantial turning in place for the leaders,
and can be described as: leaders work 1/4 emulate,
trailers work 3/4 emulate, Follow Your Leader.
%The effect of the trailers working 3/4 emulate despite having turned
%the total amount of the call is to stop them from moving ahead to be 1 and 2 in a column.

Many of the above examples result in adding rolls to existing calls,
or in working fractional twosome,
but this is not necessary.
Consider, for example, 1/2 emulate Swing Thru
(or other calls in the $n$/4 Thru family, with $n$/4 emulate).
On this call, the new centers U-Turn back toward each other,
which is counter to their rolling direction.
Calls of this ilk may or may not end up feeling good to dance!

\section{A worked example}

Here we work three examples, all involving Box Transfer.
From a given starting setup, we show three variants of the call:
emulating after 1/4, 1/2, and 3/4.

1/4 emulate Box Transfer
follows plainly from the definition,
but 1/2 emulate and 3/4 emulate
Box Transfer require breathing to create a new plane for the centers,
resulting in a Z.

\displaytwo{
  \dancer{1}{n} & \dancer{2}{s} \\
  \dancer{3}{n} & \dancer{4}{s} \\
}{starting formation for $n$/4 emulate Box Transfer}{
  \dancer{1}{w} \\
  \dancer{3}{w} \\
  \dancer{2}{e} \\
  \dancer{4}{e} \\
}{1/4 emulate Box Transfer}

\displaytwo{
  \idancer & \dancer{1}{w} \\
  \dancer{2}{e} & \dancer{3}{w} \\
  \dancer{4}{e} & \idancer
}{1/2 emulate Box Transfer}{
  \dancer{4}{e} & \dancer{2}{e} & \idancer \\
  \idancer & \dancer{3}{w} & \dancer{1}{w} \\
}{3/4 emulate Box Transfer}

\vspace{0.5cm}

Similarly, Curl Thru (which also ends in a Z) can be defined
as 1/2 emulate Crossfire.
On the other hand, Scoot Back to a Wave
is defined to end in a single quarter tag,
and is therefore {\em not} the same as 1/2 emulate Scoot Back.

% or ``Properties and Limitations''?  hm, dunno.
\section{Not exactly analogous}
\label{sec:nonanalogy}

The fractional emulate concept is analogous to fractional stable,
but some properties are different,
because the concept can change the ending position of the call.

The 3/4 emulate Box Transfer example (above) illustrates one difference:
if a call involves turning $n$/4, the $n$/4 stable
version will be equivalent to the original call, while the $n$/4
emulate version may not. The concepts differ when the call involves
changing position but not direction after the last turn (for example,
extending in the case of box transfer).

The $(n+1)/4$ emulate version, however,
will indeed be equivalent to the original call.

With fractional stable,
if a given call X is legal from a given starting setup,
then fractional stable X is also possible.
(This is also true for just plain stable, or just plain emulate.)
With fractional emulate, this is not the case!
Consider, for example: 1/2 emulate, Box Circulate twice.
In this call, people try to end up on the same spot
with the same facing direction,
so they cannot apply collision rules.

Some calls won't work well, because the position isn't well-defined
when you've turned a certain number of quarters.  Consider, for
example, zoom: the leader's position is well-defined after 1/2, but
not after 1/4 or 3/4.

Other applications might be legal, but awkward:
1/4 emulate Box Circulate, for example,
would leave the original leaders in single quarter tag spots,
and the original trailers in box spots.
It's not clear what a caller would want to do from there!


\section{Thanks!}

Thanks to Janet Chuang for discovering this concept,
by noticing that Zip the Top actually does have something to do
with other ``The Top'' calls -- contrary to Laura's complaints!
(These complaints are now rescinded.)

Thanks also to Andy Latto for the excellent example
of an awkward ending position (1/4 emulate Box Circulate),
and for a sanity check of some other ideas.

\end{document}


