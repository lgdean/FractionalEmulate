\documentclass[11pt]{article}
\usepackage{fullpage}
\usepackage{tex-squares}
%\usepackage{doublespace}
%\usepackage{graphicx}
%\usepackage{moreverb}

\title{The Fractional Emulate Concept (DRAFT)}
\author{Rachel Chasin and Laura Dean}
\date{\today}
%\markright{foo}

\begin{document}
\thispagestyle{empty}
\maketitle

%\tableofcontents

This concept is analogous to fractional stable:
do the given call until you've turned the stated fraction,
and then continue with the rest of the call, but emulate.

As with the fractional stable concept, different people may start
working emulate at different times.

Thanks to Janet Chuang for accidentally discovering the concept,
to Rachel Chasin for naming it, and to me for writing about it.

This concept came about when the author expressed frustration that Zip
the Top had no apparent relation to the other ``the top'' calls -- in
particular, nobody was casting off 3/4. 
``Well, actually, it's Fan the Top, but emulate after turning 1/4,''
Janet helpfully noted.  And I suppose it is!

Some fun equivalences:

\begin{tabular}{ll}
\hline
1/4 emulate Fan the Top   & Zip the Top \\
1/4 emulate Cast off 3/4 (from mini-wave)  & Shazam \\
1/4 emulate Box Transfer & Follow Thru and U-Turn Back (Tandem Twosome Shazam) \\
1/4 emulate Follow Your Neighbor  & same as previous :) \\
1/4 emulate Trade                          & Single Wheel \\
1/4 emulate Scoot Back  & Single Ferris Wheel (Tandem Twosome single wheel)\\
1/4 emulate Couples Trade  & Stack the Wheel \\
\hline \\
\end{tabular}

Fractional emulate can also describe a relationship between similarly named calls:

\begin{tabular}{ll}
\hline
Leads work 1/4 emulate, Follow Your Leader & Follow Your Neighbor \\
Fractal, 1/4 emulate Follow Your Leader    & Follow Your Neighbor \\
\hline \\
\end{tabular}

\section{An interesting non-analogy}

The fractional emulate concept is analogous to fractional stable,
but (at least) one property is different.
With fractional stable,
if a given call X is legal from a given starting setup,
then fractional stable X is also possible.
(This is also true for just-plain stable, or emulate.)
With fractional emulate, this is not the case!
Consider, for example: 1/2 emulate, Box Circulate twice.
In this call, people try to end up in the same spot,
with the same facing direction
(so they can't even think about taking right hands).


\section{Fractional emulate and breathing}

An interesting case with fractional emulate using fractions larger
than 1/4 arises with box transfer. 1/4 emulate Box Transfer
follows plainly from the definition,
but 1/2 emulate and 3/4 emulate
Box Transfer require breathing. For the 1/2 emulate case, after the leaders
have turned 1/2 and stop changing position, they have box circulated
one spot from their original position, and are therefore still in
adjacent planes. However, the original trailers extend and arm turn 1/2
before they stop changing positions, putting them on the center line
of the box and forcing everyone to breathe to create three planes.
This makes the ending formation a Z (or diamond spots?).
A similar Z (or quarter tag? -- we can't see why, but it feels similar
to the breathing done in Scoot Back to a Wave) is formed by
3/4 emulate Box Transfer. This marks another difference from
fractional stable: if a call involves turning $n$/4, the $n$/4 stable
version will be equivalent to the original call while the $n$/4
emulate version may not. The concepts differ when the call involves
changing position but not direction after the last turn (for example,
extending in the case of box transfer).

\displaytwo{
  \dancer{1}{n} & \dancer{2}{s} \\
  \dancer{3}{n} & \dancer{4}{s} \\
}{starting formation for $n$/4 emulate box transfer}{
  \dancer{1}{w} \\
  \dancer{3}{w} \\
  \dancer{2}{e} \\
  \dancer{4}{e} \\
}{1/4 emulate box transfer}

\displaytwo{
  \idancer & \dancer{1}{w} \\
  \dancer{2}{e} & \dancer{3}{w} \\
  \dancer{4}{e} & \idancer
}{1/2 emulate box transfer}{
  \dancer{4}{e} & \dancer{2}{e} & \idancer \\
  \idancer & \dancer{3}{w} & \dancer{1}{w} \\
}{3/4 emulate box transfer}

If the Z reasoning applies, then Curl Thru can be defined
as a 1/2 emulate Crossfire.

\section{Relations to Existing Concepts}

Many fractional emulate calls end up being ``and roll'' or ``and roll and
roll'' endings of existing calls.
One counterexample is 1/2 emulate Swing Thru
(or other calls in the $n$/4 Thru family, with $n$/4 emulate).
On this call, the new centers u-turn back toward each other.
Calls of this ilk may or may not end up feeling good to dance!

Another counter-example, of course, is seen above:
calls (such as Box Transfer) that end with walking ahead
(or sliding) rather than turning.

\section{Limitations}

Some calls won't work well, because the position isn't well-defined
when you've turned a certain number of quarters.  Consider, for
example, zoom: the leader's position is well-defined after 1/2, but
not after 1/4 or 3/4.


\end{document}


